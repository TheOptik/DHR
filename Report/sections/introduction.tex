\section{Introduction}
The aim of this project is the recreation of the functionality of the CHR library for SWI-Prolog using delimited continuation.

\subsection{Delimited Continuation}
In programmnig delimited continuation, also called delimited control, is the ability to generate reified structures for representing suspended sequential code and the capability of resuming said supensions.
Or as it is defined in~\cite{schrijvers2013delimited} ''It allows the programmer to suspend and capture the remaining part of a computation in order to resume it later''.
Since the nomenclature of the operators for creating and continuing suspensions can vary, the following section will explain the terms used throughout this report.
In SWI-Prolog the terms ''shift'' and ''reset'' are used to describe the keywords of delimited continuation, this was first proposed in~\cite{danvy1990abstracting}.

\subsubsection{shift/1}
The above mentioned definition states, that there needs to exist a way to ''suspend [...] the remaining parts of a computation [...]''.
SWI-Prolog provides a predicate named \inlineCode{shift/1} to enable this behaviour.
The documentation states that using the \inlineCode{shift/1} predicate allows the developer to ''Abandon the execution of the current goal, returning control to just after the matching reset/3 call.''~\cite{swipl:doc:shift}.


\subsubsection{reset/3}
As the definition, of delimited control, states, we need a way of ''[...]capture the remaining parts of a computation [...]''.
In SWI-Prolog this can be achieved with the \inlineCode{reset(:Goal, ?Ball, -Continuation)} predicate.
The documentation states:\\
''Call Goal. If Goal calls shift/1 and the argument of shift/1 can be unified with Ball [...] unifying Continuation with a goal that represents the continuation after shift/1.[...]''~\cite{swipl:doc:reset}\\
This fulfills the first half of our definition.
The second half states, that the posibility to ''[...] resume it later.'' has to also be given.
If we further examine the definition of the \inlineCode{reset/3} predicate it states:\\
''meta-calling Continuation completes the execution where shift left it[...]''~\cite{swipl:doc:reset}\\
Meaning that using the \inlineCode{reset/3} predicate together with the aforementioned \inlineCode{shift/1} predicate gives us a complete framework for delimited control.

\newpage