\section{Usage}
The usage of the {\dhr} library is strongly dictated by the usage of the CHR library, since one of the main goals was being a drop-in replacement for the original CHR library.
{\dhr} allows users to declare constraints, define rules, as well as query the current program.

\subsection{Constraint Declaration}
Every constraint used in a program has to be declared using the \inlineCode{chr_constraint/1} declaration.
For example a constraint consisting of the functor name \inlineCode{foo} with two arguments would be declared as follows:

\begin{lstlisting}[frame=single, caption=single constraint declaration]
:- chr_constraint foo/2.
\end{lstlisting}

Forthermore, constraint declarations may be combined via a comma-separated list, or split throughout the program, making the following examples valid.

\begin{lstlisting}[frame=single,caption=multiple constraint declarations separated by arbitrary code]
:- chr_constraint foo/2, bar/0.
% [...]
:- chr_constraint baz/4.
\end{lstlisting}


\subsection{Rule Definition}\label{sec:rule_def}
The definition of a rule consists of multiple parts, that are dependent on the type of rule.
The different types of rules and their respective definition patterns, as described in~\cite{fruhwirth2008welcome}, are as follows:
\begin{itemize}
\item Simplification rule: $Name $ @ $  H \Leftrightarrow G $ | $ B$

\hspace{100pt} [Name '@']   H '<=>' [G  '|']  B.
\item Propagation rule:\quad $Name $ @ $  H \Rightarrow G $ | $ B$

\hspace{100pt} [Name '@']   H '==>' [G  '|']  B.
\item Simpagation rule:\quad $Name $ @ $  H_1 $ {\textbackslash} $ H_2 \Leftrightarrow G $ | $ B$

\hspace{100pt} [Name '@']   $H_1$ {\textbackslash} $H_2$ '<=>' [G  '|']  B.
\end{itemize}

\noindent Since simpagation rules are a generalization of both the simplification and porpagation rules, all rules are internaly represented as simpagation rules.
If the original rule was a simplification, then $H_1$ is empty.
Should the original rule have been a propagation rule $H_2$ is empty.
In literature such a representation is refered to as generalized simpagation rule.

\subsection{Examples}
The following section is a collections of sample programs and their usages.

\subsubsection{Greatest Common Denominator (gcd)}

\begin{lstlisting}[frame=single, caption=chr program for calculating the greatest common denominator, label={lst:gcd}]
%Imports
% [...]
%Program
:- chr_constraint gcd/1.
gcd(0) <=> true.
gcd(N) \ gcd(M) <=> M >= N | NN is M - N, gcd(NN).
\end{lstlisting}
The program depicted in \ref{lst:gcd} can be queried as follows:\\
\inlineCode{?- gcd(6),gcd(9)}\\
Resulting in an output of:\\
\inlineCode{[gcd(3)]}

\subsubsection{Transitive Closure}

\begin{lstlisting}[frame=single, caption=chr program for calculating the transitive closure of a graph, label={lst:tc}]
%Imports
% [...]
%Program
:- chr_constraint edge/2, path/2.
rem_dup  @ path(X, Y) \ path(X, Y) <=> true.
path_1   @ edge(X, Y) ==> path(X, Y).
path_add @ edge(X, Y), path(Y, Z) ==> X \== Y,
                                      Y \== Z | path(X, Z).
\end{lstlisting}
The program depicted in \ref{lst:tc} can be queried as follows:\\
\inlineCode{?- edge(1, 2), edge(2, 3), edge(2, 4)}\\
Resulting in an output of:\\
\inlineCode{[edge(2, 4), edge(2, 3), edge(1, 2), path(1, 4), path(2, 4), path(1, 3), path(2, 3), path(1, 2)]}
\newpage