\section{Representations}

The implementation of {\dhr} has two central representations of the current CHR context, namely the set of all \inlineCode{chr_rule}'s that have been created by the term expansion and the \inlineCode{chr_state} that propagates the current state through the program.

\subsection{chr_rule/5}
The signature of the chr_rule facts is a follows:\\
\inlineCode{chr_rule(Active, HeadKeep, HeadRemove, Body, Guard)}.\\
It consists of the following Parts:

\begin{itemize}
\item \textbf{Active}

The active Constraint of the CHR rule.
\item \textbf{HeadKeep}

The left head part of the generalized simpagation rule, which has to be matched to the constraint store, but will not be removed from the constraint store after
\item \textbf{HeadRemove}

The right head part of the generalized simpagation rule. The matching constraints will be removed from the constraint store after the rule was applied
\item \textbf{Body}

The body of the rule, this will be executed if all the head constraints can be matched with the current constraint store and if the guard can succesfully be applied
\item \textbf{Guard}

The guard that has to be succesfully executable after the heads were matched with the current constraint store.
\end{itemize}

\subsection{chr_state/3}
The signature of the chr_rule facts is a follows:\\
\inlineCode{chr_state(ActiveConstraint, Continuations, ConstraintStore)}.\\
It consists of the following Parts:

\begin{itemize}
\item \textbf{ActiveConstraint}

The currently active constraint
\item \textbf{Continuations}

The continuation stack, represented as a conjunction of continuations.
\item \textbf{ConstraintStore}

The current constraint store, represented as a list of constraints
\end{itemize}

\newpage
